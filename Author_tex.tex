%% Author_tex.tex
%% V1.0
%% 2012/13/12
%% developed by Techset
%%
%% This file describes the coding for rsproca.cls

\documentclass[]{rsos}%%%%where rsos is the template name

%%%% *** Do not adjust lengths that control margins, column widths, etc. ***
\usepackage{gensymb}
%%%%%%%%%%% Defining Enunciations  %%%%%%%%%%%
\newtheorem{theorem}{\bf Theorem}[section]
\newtheorem{condition}{\bf Condition}[section]
\newtheorem{corollary}{\bf Corollary}[section]
%%%%%%%%%%%%%%%%%%%%%%%%%%%%%%%%%%%%%%%%%%%%%%%


\begin{document}

%%%% Article title to be placed here
\title{Insert the article title here}

\author{%%%% Author details
X. X. Nyssa Silbiger$^{1}$, X. Second author$^{2}$ and X. Third author$^{3}$}

%%%%%%%%% Insert author address here
\address{$^{1}$University of California, Irvine\\
$^{2}$Second author address\\
$^{3}$Third author address}

%%%% Subject entries to be placed here %%%%
\subject{xxxxx, xxxxx, xxxx}

%%%% Keyword entries to be placed here %%%%
\keywords{xxxx, xxxx, xxxx}

%%%% Insert corresponding author and its email address}
\corres{Nyssa Silbiger\\
\email{nyssa.silbiger@uci.edu}}

%%%% Abstract text to be placed here %%%%%%%%%%%%
\begin{abstract}
The abstract text goes here. The abstract text goes here. The abstract text goes here. The abstract text goes here.
The abstract text goes here. The abstract text goes here. The abstract text goes here. The abstract text goes here.
\end{abstract}
%%%%%%%%%%%%%%%%%%%%%%%%%%%

%%%%%%%%%% Insert the texts which can accomdate on firstpage in the tag "fmtext" %%%%%

\begin{fmtext}
\section{Introduction}
%%%% Insert A head here

Introduction goes here \\
x\\
x\\
x\\
x\\
x\\
x\\
x\\
x\\
x\\
x\\
x\\
x\\
x\\
x\\
x\\
x\\
x\\
x\\


\end{fmtext}
%%%%%%%%%%%%%%% End of first page %%%%%%%%%%%%%%%%%%%%%

\maketitle

\section{Materials and Methods}
\subsection{Sample Collection}
 Coral, macroalgae, rubble, and sand samples were collected from a fringing reef just adjacent to the Hawai`i Institute of Marine Biology (21.435$^{\circ}$, -157.787$^{\circ}$) between October 12 and 16, 2015. We collected thirty-six coral fragments (nubbins) from three individual colonies (n=12 per colony) of the two dominant coral species in  K\={a}ne`ohe Bay, Hawai`i (\textit{Porites compressa} and \textit{Montipora caiptata}). Nubbins were collected from visibly healthy colonies with a hammer and chisel between a depth of 4 and 7 m. Rubble from dead \textit{Porites} sp. skeleton was collected haphazardly at the same depths as the live coral. Macroalgae (\textit{Gracilaria salicornia}) and sand were collected between X and X m depth. Macroalgae was collected haphazardly from the reef and pooled into a collecting bag. Thirty-six sand cores were collected from the top 3 cm of aerobic reef sediment and placed them into glass petri dishes. \\
\indent After collection, samples were immediately pooled by group into aquaria with flow through seawater from K\={a}ne`ohe Bay and natural light conditions for XX days. Samples were then buoyant weighed (CITE BW PAPER) and evenly divided into 36 groups. Three coral nubbins (one per colony) from each species were adhered to plastic egg crates with All Fix putty for a total of 72 crates. Individual \textit{P. compressa} (weight +/- SE) and \textit{M. capitata} (weight +/- SE) were cable tied together for a total of 36 coral replicates. Both the coral rubble (weight +/- SE) and macroalage (weight +/- SE) were distributed into 36 vexar mesh containers and the sand remained in the 36 petri dishes. At the end of the experiment, all samples were re-buoyant weighed and then ashed (how or cite how) to determine growth rates and organic content, respectively.
 

\subsection{Experimental Set-up}
Four replicates of each reef constituent were placed into individual, clean polycarbonate aquaria (6L total volume). Each experimental aquarium was affixed with an upper spigot drain to hold the water level constant at 5L and contained a Rio plus 50 aqua pump (320 l h$^{-1}$) to ensure that the water within the aquarium was well-mixed. The 36 aquaria were divided into three 1300L incubation tanks (12 per tank) with natural flow-through seawater in order to maintain a constant temperature. Incubation tanks were outfitted with HOBOs (type) and PAR (type) loggers to monitor temperature and light, respectively (what was the frequency). Each incubation tank held one nutrient by reef constituent treatment (3 nutrient levels x 4 reef constituents) for a fully blocked design (Fig X).  \\
\indent Incoming seawater was filtered through a sand filter followed by a 20 $\mu$M  (GF/F??) filter before entering our experimental aquaria. The nutrient levels were maintained by mixing source water from K\={a}ne`ohe Bay with a concentrated nutrient mix in a 20L pre-cleaned carboy every other day (IS this RIGHT?). The nutrient mix was a frozen stock containing a ratio of 3:1 potassium nitrate: potassium phosphate and was stored at ambient temperatures in the dark. Both the source water and the nutrient mixture were pumped and pre-mixed into separate aquaria (hereafter called header aquaria) using multi-channel peristaltic pumps before entering the experimental aquaria. Each header aquaria contained a  Rio plus 90 aqua pump (320 l h$^{-1}$) to ensure that the nutrients were mixed into solution. The nutrients were pumped through platinum cured silica tubes resulting in a residence time of approximately 2 h and 6 h in the header and experimental aquaria, respectively. Three nutrient levels were maintained throughout the experiment: ambient (XX $\mu$M NO$_{3}^{-}$ and $\mu$M PO$_{4}^{3-}$ ), medium (XX $\mu$M NO$_{3}^{-}$ and $\mu$M PO$_{4}^{3-}$), and high (XX $\mu$M NO$_{3}^{-}$ and $\mu$M PO$_{4}^{3-}$).  \\
\indent Experimental conditions were maintained for a 6-week acclimation period. All sample storage containers (i.e. vexar and plastic crates) were carefully cleaned and all 36 experimental aquaria were replaced with new, clean aquaria weekly. During weekly cleanings, the aquaria were randomly rotated both within and across incubation tanks to account for differences in light. Nutrient samples were collected biweekly throughout the experiment in all experimental and header aquaria to ensure that the nutrient conditions were maintained.

\subsection{Metabolism Experiments}
We ran two 24 hour metabolism experiments to determine the effect of nutrient on both individual reef constituents and a mixed communities. The first experiment was run immediately after the 6-week acclimation period. We then randomly mixed the communities such that each aquarium had one replicate of each organism, allowed them to acclimate to their new tank conditions for 24 hours and then repeated the metabolism experiment. During the two 24 hour experiments, all header and experimental aquaria were temporarily covered with plastic wrap to reduce off-gassing.\\
\indent During each metabolism experiment, we collected water samples for total alkalinity ($A_{T}$) and nutrients ($NO_{3}^{-}$ + $NO_{2}^{-}$, $PO_{4}^{3-}$), and measured temperature and pH \textit{in situ} with sensors every 4 hours over a 24 hour period (7 sampling periods) in all header and experimental aquaria.  
  
\subsubsection{pH and temperature}
pH measurements were taken with an Orion ROSS Ultra pH/ATC Triode glass electrode \textit{in situ} in mV. The pH electrode was calibrated against a Tris buffer of known pH from Andrew Dickson's laboratory at the Scripps Institution of Oceanography following Dickson SOPXX (cite). pH was calculated on the total scale (pH$_{Tot}$) from mV and temperature measured on a BLAH \textit{in situ}. 
\subsubsection{$A_{T}$ and nutrients}
All water samples were collected in acid washed bottles, each rinsed three times with sample water.  $A_{T}$ samples were collected in 250 ml Nalgene bottles and immediately preserved with 100 $\mu L$ of 50\% saturated $HgCl_{2}$. Total alkalinity was analyzed using open cell potentiometric titrations on a Mettler T50 autotitrator \cite{Dickson2007}. A certified reference material (CRM -- Reference Material for Oceanic CO$_2$ Measurements, A. Dickson, Scripps Institution of Oceanography) was run at the beginning of each sample set. The accuracy of the titrator was always better than 1\% off from the standard, and TA measurements were corrected for these deviations. Nutrient samples were collected in 60 ml plastic syringes, filtered through pre-combusted GF/F (0.7 $\mu m$) filters, transferred into 50 ml plastic centrifuge tubes, and frozen until analysis. Nutrient samples were processed on a Seal Analytical AA3 HR Nutrient Analyzer at the UH SOEST Lab for Analytical Chemistry for $NO_{3}^{-}$ + $NO_{2}^{-}$ and $PO_{4}^{3-}$.  


\subsubsection{ Metabolic calculations}
We used the total alkalinity anomaly technique to calculate net ecosystem calcification (NEC) and net community production (NCP) rates for individual reef constituents and mixed reef communities. Total alkalinity was corrected for dissolved inorganic nitrogen and phosphate in the header aquaria to account for their small contributions to the acid-base system \cite{wolf2007total}.NEC rates ($\mu$mol CaCO$_3$ g$^{-1}$ h$^{-1}$) were calculated using the following equation: 
\begin{align}
\label{1.1}
\begin{split}
NEC  = \frac{\Delta A_{T} \cdot V \cdot \sigma}{2\cdot t\cdot  m}
\end{split}
\end{align}

$\Delta A_{T}$  ($\mu$mol kg$^{-1}$) is the difference in total alkalinity between the header and experimental aquaria, $V$  (cm$^{3}$) is the volume of water in the experimental aquaria, $\sigma$ is the density of seawater (1.023 g cm$^{-3}$), $t$  (h) is the residence time of the experimental aquaria, and $m$ (g) is the ash free dry weight (AFDW) of the samples. $\Delta A_{T}$ was divided by 2 because 1 mol of CaCO$_3$ is produced for every 2 mols of $A_{T}$. NEC rates were then divided by 1000 to yield rates in units $\mu$mol $CaCO_3$ g$^{-1}$ h$^{-1}$. NEC rates were averaged across daylight hours, dark hours, and over the entire 24 h cycle to obtain daytime calcification, nighttime calcification, and net calcification over 24 h, respectively in each aquaria. \\
\indent NCP rates ($\mu$mol C g$^{-1}$ h$^{-1}$) were calculated using the following equation:
\begin{align}
\label{1.2}
\begin{split}
NCP  = \frac{\Delta DIC \cdot V \cdot \sigma}{t\cdot  m}-NEC
\end{split}
\end{align}

$\Delta$DIC is the difference in dissolved inorganic carbon ($\mu$mol kg$^{-1}$) between the header and experimental aquaria. Because all header and experimental aquaria were covered during the sampling period we assumed off-gassing was negligible and thus did not calculate FCO$_2$ rates. DIC was calculated from $A_{T}$ and pH using seacarb (CITE and name parameters). NCP rates were averaged over the 24 h sampling period to obtain net production rates for each aquaria. Respiration rates (R) were calculated by averaging NCP rates during dark hours and gross community production (GCP) rates were calculated as R + NCP during daylight hours.

\subsection{Data Analysis}
stats

\section{Results}
\subsection{Change in buoyant weight over 6-weeks}
Over the 6-week acclimation period we saw an x\% change in biomass in......xx was significant.\\
\subsection{Metabolic response}
Both calcification and production rates were influenced by nutrient addition. NEC and daytime NEC both significantly decreased in response to nutrient addition in all reef constituents and the mixed community (Figure/Table). Algae, rubble, sand, and the mixed community also switched from net calcifying to net dissolving with increasing nutrients over the 24 h cycle (Figure/Table). Nighttime NEC was reduced (or dissolution increased) with added nutrients in all reef constituents, although the effect of nutrients on nighttime NEC in rubble was not statistically significant(Figure/Table). \\
\indent  NCP was unaffected by nutrients addition in all reef constituents (Figure/table). GCP, however, increased with nutrient addition in all reef constituents, but was only statistically significant in the coral, sand, and the mixed community. Respiration also increased with nutrients in coral, rubble, algae, and sand, but not in the mixed community (Figure/table). 

\subsection{Biological feedbacks}
pH increased during the day and decreased at night in all experimental aquaria (figure) as a result of daytime photosynthesis (CO$_2$ absorption) and nighttime respiration (CO$_2$ release). The community composition had a substantial effect on the local pH environment with coral having the strongest ($\Delta$ 0.08 $\pm$ 0.012 pH units; mean $\pm$ in ambient treatment) and sand having the weakest effect on pH($\Delta$ 0.013 $\pm$ 0.001 pH units; mean $\pm$ in ambient treatment) (figure X). The strong positive relationship between NCP and pH (F = $2049_{2,501}$ p<0.001, r$^2$ = 0.89; Figure) indicates that these differences in pH were driven by the different NCP rates in reef constituents (cite NCP figure). However, we saw weak evidence of biological feedbacks in response to nutrient addition. pH variability did increase slightly with nutrient addition, but not significantly and this result mirrors the effect of nutrients on NCP (Figure/table). \\
\indent Aragonite saturation state ($\Omega_{arag}$) and NEC were positively correlated in the ambient treatments for coral, rubble, sand, and the mixed community (Figure/Table). However, nutrient addition significantly weakened this relationship in all reef constituents, and, in the sand, caused a complete disassociation between $\Omega_{arag}$ and NEC.

\section{Discussion}
There is a long history of research on the impacts of nutrients on coral reef community members (cite reviews). However, this is the first study to use a chemostat (supplying a constant nutrient concentration over a 6-week period), test the impacts of nutrients on individuals and a mixed community, and also test for biological feedbacks in response to nutrient addition.

We saw an effect of nutrients on NEC/NCP.  Other studies have too and here is how.

While we did see a slight increase in the pH variability between nutrient treatments, it was not statistically significant indicating weak biological feedbacks in response to nutrients. This is also evident by the lack of effect of nutrients on NCP. However, we did see a significant effect of nutrients on calcification and, further, a weakening of the relationship between NEC and $\Omega_{arag}$ in all community members. These results suggest that the negative effect of nutrients on coral reef communities is likely a physiological response to nutrients rather than in response to a shifting pH environment. Studies have shown in corals that nutrient addition can cause competition between Symbiodinium and their coral host, lowering calcification (CITE) (from mutualists to parasites).  Other examples, why is this true for sediment (CACO3 sediment (cite true for Hawaii) and cyanos/diatoms... other calcifiers in water column cocolithiphores? micro inverts in sand like crabs) and rubble (CCA/invert calcifiers and bioeroders and fleshy algae)...  Algae (had bryozoans on it that calcify), mixed community (everything). 

Other thoughts.... nutrients can increase bioerosion in rubble (could be why we see lower NEC with nut in rubble treatment)

I have notes that says nutrients reduces bicarbonate availability.... look this up.

Shifting community composition will shift the pH environment,. 

\section{Conclusion}
The conclusion text goes here.




\section{Figures \& Tables}

The output for figure is:

\begin{figure}[!h]
%\centering\includegraphics[width=2.5in]{xxxxxx.eps}
%%% where xxxxxx name represents "figurename.eps"
\caption{Insert figure caption here}
\label{fig_sim}
\end{figure}

\vspace*{-10pt}

\noindent The output for table is:

\begin{table}[!h]
\caption{An Example of a Table}%%%Table caption goes here
\label{table_example}
\begin{tabular}{llll}%%%The number of columns has to be defined here
\hline
date &Dutch policy &date &European policy \\
\hline
1988 &Memorandum Prevention &1985 &European Directive (85/339) \\
1991--1997 &{\bf Packaging Covenant I} & & \\
1994 &Law Environmental Management &1994 &European Directive (94/62) \\
1997 &Agreement Packaging and Packaging Waste & & \\
1998--2002 &{\bf Packaging Covenant II} & & \\
2003--2005 &{\bf Packaging Covenant III} & & \\
2006--2007 &{\bf Decree on Packaging and paper} & & \\\hline
\end{tabular}
\end{table}%%%End of the table


\section*{Acknowledgment}

Insert the Acknowledgment text here.


%%%%%%%%%% Insert bibliography here %%%%%%%%%%%%%%

%\begin{thebibliography}{9}
%\end{thebibliography}

\bibliography{Accretion_erosion.bib}
\bibliographystyle{prsb}
\end{document}